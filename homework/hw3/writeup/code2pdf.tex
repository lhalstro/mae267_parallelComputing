
\documentclass[twocolumn,10pt]{asme2ej}

\usepackage{epsfig} %% for loading postscript figures
\usepackage{listings}
\usepackage{amsmath}
\usepackage{graphicx}
\usepackage{grffile}
\usepackage{pdfpages}
\usepackage{algpseudocode}
\usepackage{courier}
\usepackage{tikz}
\newcommand*\circled[1]{\tikz[baseline=(char.base)]{
            \node[shape=circle,draw,inner sep=2pt] (char) {#1};}}
%\usepackage{multicol}


% Custom colors
\usepackage{color}
\usepackage{listings}
\usepackage{framed}
\usepackage{caption}
\usepackage{bm}
\captionsetup[lstlisting]{font={small,tt}}

\definecolor{mygreen}{rgb}{0,0.6,0}
\definecolor{mygray}{rgb}{0.5,0.5,0.5}
\definecolor{mymauve}{rgb}{0.58,0,0.82}

\lstset{ %
  backgroundcolor=\color{white},   % choose the background color; you must add \usepackage{color} or \usepackage{xcolor}
  basicstyle=\ttfamily\footnotesize, % the size of the fonts that are used for the code
  breakatwhitespace=false,         % sets if automatic breaks should only happen at whitespace
  % breaklines=true,                 % sets automatic line breaking
  captionpos=b,                    % sets the caption-position to bottom
  commentstyle=\color{mygreen},    % comment style
  deletekeywords={...},            % if you want to delete keywords from the given language
  escapeinside={\%*}{*)},          % if you want to add LaTeX within your code
  extendedchars=true,              % lets you use non-ASCII characters; for 8-bits encodings only, does not work with UTF-8
  frame=single,                    % adds a frame around the code
  keepspaces=true,                 % keeps spaces in text, useful for keeping indentation of code (possibly needs columns=flexible)
  columns=flexible,
  keywordstyle=\color{blue},       % keyword style
  language=Python,                 % the language of the code
  morekeywords={*,...},            % if you want to add more keywords to the set
  numbers=left,                    % where to put the line-numbers; possible values are (none, left, right)
  numbersep=5pt,                   % how far the line-numbers are from the code
  numberstyle=\tiny\color{mygray}, % the style that is used for the line-numbers
  rulecolor=\color{black},         % if not set, the frame-color may be changed on line-breaks within not-black text (e.g. comments (green here))
  showspaces=false,                % show spaces everywhere adding particular underscores; it overrides 'showstringspaces'
  showstringspaces=false,          % underline spaces within strings only
  showtabs=false,                  % show tabs within strings adding particular underscores
  stepnumber=1,                    % the step between two line-numbers. If it's 1, each line will be numbered
  stringstyle=\color{mymauve},     % string literal style
  tabsize=4,                       % sets default tabsize to 2 spaces
}

\begin{document}
\onecolumn
\appendix       %%% starting appendix
\section*{MAE 267 - Homework 3}

\noindent Logan Halstrom

\noindent 22 October 2015\\

Integration and timing results for runs with various numbers of processors (Table~\ref{table}):

\vspace{-1.5em}
\begin{table}[htb]
\begin{center}
\caption{Timing and Solver Results}
\begin{tabular}{|c | c c c c|}
\hline
\textbf{Number of CPUs} & \textbf{Wall Time (s)} & \textbf{Wall Time Per Processor (s)} & \textbf{$\pi$ Result} & \textbf{$\pi$ Error} \\
\hline
\textbf{1} & 7.2956e-05 & 7.2956e-05 & 3.14159265358979356009 & 4.4409e-16 \\
\textbf{2} & 1.7095e-04 & 8.5473e-05 & 3.14159265358979400418 & 8.8818e-16 \\
\textbf{4} & 0.0013 & 3.3724e-04 & 3.14159265358979311600 & 0.0000e+00 \\
\textbf{8} & 0.0011 & 1.3199e-04 & 3.14159265358979267191 & -4.4409e-16 \\
\hline
\end{tabular}
\label{table}
\end{center}
\end{table}
\vspace{-2em}
Somehow, the solution for $\pi$ with 4 processors matched Python's stored value exactly (indicated by zero error for the NCPUS=4 case.)\\
\\
\\
Solution wall time results as a function of number of processors used (Figure~\ref{walltimes}:

\vspace{-1.5em}
\begin{figure}[htb]
\begin{center}
\includegraphics[width=0.45\textwidth]{../Results/walltimes_simp.png}
\includegraphics[width=0.45\textwidth]{../Results/walltimes_simplong.png}
\caption{Total wall times for solving $\pi$ using Simpson's rule, with 1-8 CPUs on the left and 1-128 on the right}
\label{walltimes}
\end{center}
\end{figure}
\vspace{-2em}

In general, increasing numbers of processors tend to increase the total wall time as more communication time is required for a given solution.

% \clearpage

\section*{Parallel Wrapper Code}
\lstinputlisting[language=Fortran]{../calcpip.f}

\section*{Simpson's Rule Subroutine}
\lstinputlisting[language=Fortran]{../simp.f}

% \clearpage

\section*{Sample Output}
\lstinputlisting[language={}]{../Results/simp_i256_np8.dat}



\clearpage
\section*{Plotting Code}
\lstinputlisting[language=Python]{../plotTimes.py}

\end{document}
