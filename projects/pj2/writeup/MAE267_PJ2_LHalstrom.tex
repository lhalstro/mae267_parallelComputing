
%%% use twocolumn and 10pt options with the asme2ej format
\documentclass[twocolumn,10pt]{asme2ej}

\usepackage{epsfig} %% for loading postscript figures
\usepackage{listings}
\usepackage{amsmath}
\usepackage{graphicx}
\usepackage{grffile}
\usepackage{pdfpages}
\usepackage{algpseudocode}
\usepackage{courier}
\usepackage{tikz}
\newcommand*\circled[1]{\tikz[baseline=(char.base)]{
            \node[shape=circle,draw,inner sep=2pt] (char) {#1};}}
%\usepackage{multicol}


% Custom colors
\usepackage{color}
\usepackage{listings}
\usepackage{framed}
\usepackage{caption}
\usepackage{bm}
\captionsetup[lstlisting]{font={small,tt}}

\definecolor{mygreen}{rgb}{0,0.6,0}
\definecolor{mygray}{rgb}{0.5,0.5,0.5}
\definecolor{mymauve}{rgb}{0.58,0,0.82}

\lstset{ %
  backgroundcolor=\color{white},   % choose the background color; you must add \usepackage{color} or \usepackage{xcolor}
  basicstyle=\ttfamily\footnotesize, % the size of the fonts that are used for the code
  breakatwhitespace=false,         % sets if automatic breaks should only happen at whitespace
  % breaklines=true,                 % sets automatic line breaking
  captionpos=b,                    % sets the caption-position to bottom
  commentstyle=\color{mygreen},    % comment style
  deletekeywords={...},            % if you want to delete keywords from the given language
  escapeinside={\%*}{*)},          % if you want to add LaTeX within your code
  extendedchars=true,              % lets you use non-ASCII characters; for 8-bits encodings only, does not work with UTF-8
  frame=single,                    % adds a frame around the code
  keepspaces=true,                 % keeps spaces in text, useful for keeping indentation of code (possibly needs columns=flexible)
  columns=flexible,
  keywordstyle=\color{blue},       % keyword style
  language=Python,                 % the language of the code
  morekeywords={*,...},            % if you want to add more keywords to the set
  numbers=left,                    % where to put the line-numbers; possible values are (none, left, right)
  numbersep=5pt,                   % how far the line-numbers are from the code
  numberstyle=\tiny\color{mygray}, % the style that is used for the line-numbers
  rulecolor=\color{black},         % if not set, the frame-color may be changed on line-breaks within not-black text (e.g. comments (green here))
  showspaces=false,                % show spaces everywhere adding particular underscores; it overrides 'showstringspaces'
  showstringspaces=false,          % underline spaces within strings only
  showtabs=false,                  % show tabs within strings adding particular underscores
  stepnumber=1,                    % the step between two line-numbers. If it's 1, each line will be numbered
  stringstyle=\color{mymauve},     % string literal style
  tabsize=4,                       % sets default tabsize to 2 spaces
}



%% The class has several options
%  onecolumn/twocolumn - format for one or two columns per page
%  10pt/11pt/12pt - use 10, 11, or 12 point font
%  oneside/twoside - format for oneside/twosided printing
%  final/draft - format for final/draft copy
%  cleanfoot - take out copyright info in footer leave page number
%  cleanhead - take out the conference banner on the title page
%  titlepage/notitlepage - put in titlepage or leave out titlepage
%
%% The default is oneside, onecolumn, 10pt, final

\title{MAE 267 -- Project 2\\Serial, Multi-Block, Finite-Volume Methods\\For Solving 2D Heat Conduction}

%%% first author
\author{Logan Halstrom
    \affiliation{
	PhD Graduate Student Researcher\\
	Center for Human/Robot/Vehicle Integration and Performance\\
	Department of Mechanical and Aerospace Engineering\\
	University of California, Davis\\
	Davis, California 95616\\
    Email: ldhalstrom@ucdavis.edu
    }
}

\begin{document}
\maketitle

%%%%%%%%%%%%%%%%%%%%%%%%%%%%%%%%%%%%%%%%%%%%%%%%%%%%%%%%%%%%%%%%%%%%%%
\section{Statement of Problem}

This analysis details the solution of the steady-state temperature distribution on a 1m x 1m block of steel with Dirichlet boundary conditions (Eqn~\ref{dirichlet}).  Solutions were performed on square, non-uniform grids rotatated in the positive z-direction by $rot=30^o$.  Two grids of 101x101 points and 501x501 points were used to solve the equation of heat transfer.  Temperature was uniformly initialized to a value of 3.5 and the solution was iterated until the maximum residual found was less than $1.0x10^{-5}$.  The equation for heat conduction (Eqn~\ref{heat}) was solved using an explicit, node-centered, finite-volume scheme, with an alternative distributive scheme for the second-derivative operator.  Steady-state temperature distribution was saved in a PLOT3D unformatted file, and CPU wall time of the solver was recorded.

%%%%%%%%%%%%%%%%%%%%%%%%%%%%%%%%%%%%%%%%%%%%%%%%%%%%%%%%%%%%%%%%%%%%%%%%
\section{Equations and Algorithms}

The solver developed for this analysis utilizes a finite-volume numerical solution method to solve the transient heat conduction equation (Eqn~\ref{heat}).

\begin{equation}
\begin{split}
\rho c_p \frac{\partial T}{\partial t} =
    k \left[ \frac{\partial^2 T}{\partial x^2}
    + \frac{\partial^2 T}{\partial y^2} \right]
\end{split}
\label{heat}
\end{equation}

\noindent The solution is initialized with the Dirichlet boundary conditions (Eqn~\ref{dirichlet}).

\begin{equation}
\begin{split}
T = &\left\{ \begin{array}{lll}
    \mbox{$5.0 \left[ \sin\left( \pi x_p \right) + 1.0 \right]$} & \mbox{for } &j = j_{max} \\
    \mbox{$\left| \cos\left( \pi x_p \right)\right|+1.0$} & \mbox{for } &j = 0 \\
    \mbox{$3.0 y_p + 2.0$} & \mbox{for } &i = 0, \, i_{max}
     \end{array} \right.
\end{split}
\label{dirichlet}
\end{equation}

\noindent Grids were generated according to the following (Eqn~\ref{grideqn})

\begin{equation}
\begin{split}
   rot &= 30.0 \frac{\pi}{180.0} \\
   x_p &= \cos \left[ 0.5\pi \frac{i_{max}-i}{i_{max}-1} \right] \\
   y_p &= \cos \left[ 0.5\pi \frac{j_{max}-j}{j_{max}-1} \right] \\
x(i,j) &= x_p \cos(rot) + (1.0 - y_p) \sin(rot) \\
y(i,j) &= y_p \cos(rot) + x_p \sin(rot)
\end{split}
\label{grideqn}
\end{equation}

\noindent To solve Eqn~\ref{heat} numerically, the equation is discretized according to a node-centered finite-volume scheme, where first-derivatives at the nodes are found using Green's theorem integrating around the secondary control volumes.  Trapezoidal, counter-clockwise integration for the first-derivative in the x-direction is achieved with Eqn~\ref{FVx1st}.

\begin{equation}
\begin{split}
\frac{\partial T}{\partial x} = \frac{1}{2Vol_{i+\frac{1}{2},j+\frac{1}{2}}}
    \left[ \left(T_{i+1,j} + T_{i+1,j+1} \right)Ayi_{i+1,j} \right. \\
    \left. - \left(T_{i,j} + T_{i,j+1} \right)Ayi_{i,j} \right. \\
    \left. - \left(T_{i,j+1} + T_{i+1,j+1} \right)Ayi_{i,j+1} \right. \\
    \left. - \left(T_{i,j} + T_{i+1,j} \right)Ayi_{i,j} \right]
\end{split}
\label{FVx1st}
\end{equation}

\noindent A similar scheme is used to find the first-derivative in the y-direction.

%%%%%%%%%%%%%%%%%%%%%%%%%%%%%%%%%%%%%%%%%%%%%%%%%%%%%%%%%%%%%%%%%%%%%%
\section{Results and Discussion}

Both grids used in this analysis were non-uniformly distributed according to the same function and can be observed in Figs~\ref{grid101} and ~\ref{grid501}

%%\vspace{-2em}
\begin{figure}[htb]
\begin{center}
\includegraphics[width=0.5\textwidth]{../results/101/Grid.png}
\caption{101x101 point mesh}
\label{grid101}
\end{center}
\end{figure}
%%\vspace{-2em}

%%\vspace{-2em}
\begin{figure}[htb]
\begin{center}
\includegraphics[width=0.5\textwidth]{../results/501/Grid.png}
\caption{501x501 point mesh}
\label{grid501}
\end{center}
\end{figure}
%%\vspace{-2em}

%%\vspace{-2em}
\begin{figure}[htb]
\begin{center}
\includegraphics[width=0.5\textwidth]{../results/101/Temperature.png}
\caption{Steady-state temperature distribution on a 101x101 mesh}
\label{temp101}
\end{center}
\end{figure}
%%\vspace{-2em}

%%\vspace{-2em}
\begin{figure}[htb]
\begin{center}
\includegraphics[width=0.5\textwidth]{../results/501/Temperature.png}
\caption{Steady-state temperature distribution on a 501x501 mesh}
\label{temp501}
\end{center}
\end{figure}
%%\vspace{-2em}

\noindent It can be seen that the mesh becomes more refined nearer $i_{max}$ and $j_{max}$, and that the 501x501 point mesh is significantly more dense than its counterpart, which resulted in much longer wall times for solutions.

Figs~\ref{temp101} and ~\ref{temp501} show the steady-state temperature distribution on the steel plate for each mesh.  Very little difference is apparent, with the 501x501 mesh being slightly more dissipative near the hot/cold boundaries.

%%%%%%%%%%%%%%%%%%%%%%%%%%%%%%%%%%%%%%%%%%%%%%%%%%%%%%%%%%%%%%%%%%%%%%%%
\section{Conclusion}

This project has produced a functional algorithm for solving steady-state heat conduction in serial.  Though CPU wall time of the 101x101 point grid was minimal (16.88 seconds), significant wall time was required to converge the solution for the 501x501 point grid (4763 seconds).  See Appendix A for more run parameter output.  Wall time could be reduced by parallelizing the code.

%%%%%%%%%%%%%%%%%%%%%%%%%%%%%%%%%%%%%%%%%%%%%%%%%%%%%%%%%%%%%%%%%%%%%%%%
% \section{References}
% \begin{description}
% \item[] 1. Bogard, D.G., Teiderman, W.G. ``Burst detection with single-point velocity measurements'', \emph{Journal of Fluid Mechanics}, 162:389-413, 1986.
% \end{description}








%%%%%%%%%%%%%%%%%%%%%%%%%%%%%%%%%%%%%%%%%%%%%%%%%%%%%%%%%%%%%%%
% \clearpage
\onecolumn
\appendix       %%% starting appendix
\section*{Appendix A: Sample Output}
\lstinputlisting[caption=Sample output for 101x101 grid solution, language={}]{../results/101/SolnInfo.dat}
\lstinputlisting[caption=Sample output for 501x501 grid solution, language={}]{../results/501/SolnInfo.dat}

\section*{Appendix B: Source Code}

\lstinputlisting[caption=Wrapper program for solution of 2D heat conduction, language=Fortran]{../main.f90}
\lstinputlisting[caption=Main subroutines for solver (initialization/solution/output), language=Fortran]{../subroutines.f90}
\clearpage
\lstinputlisting[caption=Modules used by solver, language=Fortran]{../modules.f90}
\lstinputlisting[caption=PLOT3D file output module (compatible with ParaView), language=Fortran]{../plot3D_module.f90}


%%%%%%%%%%%%%%%%%%%%%%%%%%%%%%%%%%%%%%%%%%%%%%%%%%%%%%%%%%%%%%%%%%%%%%
%\clearpage


%%%%%%%%%%%%%%%%%%%%%%%%%%%%%%%%%%%%%%%%%%%%%%%%%%%%%%%%%%%%%%%%%%%%%%
% The bibliography is stored in an external database file
% in the BibTeX format (file_name.bib).  The bibliography is
% created by the following command and it will appear in this
% position in the document. You may, of course, create your
% own bibliography by using thebibliography environment as in
%
% \begin{thebibliography}{12}
% ...
% \bibitem{itemreference} D. E. Knudsen.
% {\em 1966 World Bnus Almanac.}
% {Permafrost Press, Novosibirsk.}
% ...
% \end{thebibliography}

% Here's where you specify the bibliography style file.
% The full file name for the bibliography style file
% used for an ASME paper is asmems4.bst.
%\bibliographystyle{asmems4}

% Here's where you specify the bibliography database file.
% The full file name of the bibliography database for this
% article is asme2e.bib. The name for your database is up
% to you.
%\bibliography{asme2e}

%%%%%%%%%%%%%%%%%%%%%%%%%%%%%%%%%%%%%%%%%%%%%%%%%%%%%%%%%%%%%%%%%%%%%%


\end{document}
